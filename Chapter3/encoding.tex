In this chapter we present the encoding of the elevator constraints,
and how different choices can impact the performance of the design
task. First, we present the baseline heuristic method and some preliminary
approaches dealing with Genetic Algorithms (GAs). Then, we elaborate on 
lessons learned in four main aspects of the encoding, both from a general
point of view by discussing the employ of different kinds of constraints, 
and from a specific point of view by discussing how arithmetic and 
optimization techniques can handle the problem --- leading to the CSP-based
variations of \liftcreate.

In the following chapters, we use a number of variants of \liftcreate{}
while referring to the algorithm considered:
\begin{itemize}
	\item \liftcreatehr{} is the current heuristic-based engine
	\item \liftcreatebf{} is a brute-force version used as a baseline
	\item \liftcreatega{} is the Genetic Algorithm-based experiment
	\item \liftcreaters{} is a random-sampling version used as a 
		baseline w.r.t. \liftcreatega
	\item \liftcreatesmt{} is the SMT-based experiment
	\item \liftcreatecp{} is the CP-based experiment
\end{itemize}

\section{Encoding strategies}
\label{sec:solutions}

Due to the specific features of \liftcreate{} versions, there are some
differences in how configurations are generated and results are
presented, differences that we describe in the following and that we
tried to harmonize as much as possible in order to make our experimental
comparison meaningful. \liftcreatehr{} produces at most one solution --- 
supposedly the ``best'' one according to the heuristics --- for each 
\emph{prototype}, i.e., a pair comprised of a door and a car frame which 
together fit the shaft. Therefore, for each given setup, \liftcreatehr{} 
produces as many solutions as there are prototypes for which a
solution exists, where a solution features a specific placement 
of car frame and doors. The three versions of \liftcreate{} based on 
search in the space of configurations, namely \liftcreatebf{},
\liftcreaters{} and \liftcreatega{}, feature a common data flow
implementing the following phases:  
\begin{itemize}
	\item \emph{Prototype generation}: amounts to list all prototypes,
	pruning up-front those which cannot fit the given shaft.
	\item \emph{Expansion}: given a prototype, potential
	configurations are explored by attempting to place the car frame 
	and the doors inside the shaft within the allowable ranges:
	the results of this phase are  \emph{early designs}.
	\item \emph{Design validation}: given an early design, a number of 
	checks is performed in order to validate the corresponding 
	configuration and declare it feasible or not.
\end{itemize}
Specifically, the validation phase must guarantee that every constraint 
of Section~\ref{sec:model_constraints} is satisfied. The difference
among \liftcreatebf{}, \liftcreaters{} and \liftcreatega{} lies in the
expansion phase: exhaustive search for \liftcreatebf{}, random
sampling for \liftcreaters{} and genetic algorithms for
\liftcreatega{}. Since these versions may produce many feasible
configurations for each prototype, whereas \liftcreatehr{} outputs
only one, we \emph{cluster} configurations after the validation
phase. In more details, the set of valid placements for each given
prototype is clustered around a representative, in order to make the
comparison with \liftcreatehr{} possible. Finally, both in \liftcreatecp{}
and \liftcreatesmt{} the prototype expansion phase is replaced by the 
CP/SMT encoding of the constraints introduced in 
Section~\ref{sec:model_constraints}, where the choice of components is 
restricted to those that can fit the shaft. The subsequent phases of
expansion and validation are merged in the search for a solution by 
the solvers. If a cost function to drive the search towards ``optimal''
designs is supplied, then \liftcreatecp{} and \liftcreatesmt{} output 
exactly one configuration for each given setup.

\subsection{Custom heuristics: \liftcreatehr{}}
\label{sec:baseline}

\begin{algorithm}[t]
	\caption{Heuristic algorithm for \liftcreatehr{}}
	\label{alg:hr}
	\small
	\begin{algorithmic}[1]
		\Function{compute\_designs}
		{$w_{\mbox{\emph{shaft}}}$, $d_{\mbox{\emph{shaft}}}$, \emph{D} = \emph{List}$<$\emph{Door}$>$, 
			\emph{F} = \emph{List}$<$\emph{CarFrame}$>$}
		\State \emph{output} = $[\:]$ \Comment{Design list,
			initially empty}
		\State $S_{\mbox{\emph{prev}}} = 0$ \Comment{Car surface initialization}
		\State \textsc{order\_by\_dcr}(\emph{F})
		\State \textsc{order\_by\_opening}(\emph{D})
		\For{$i = 1:F.length$}
		\State $S_{\mbox{\emph{cur}}}$ =
		\textsc{compute\_car\_surface}(\emph{F[i]}, $w_{\mbox{\emph{shaft}}}$, $d_{\mbox{\emph{shaft}}}$)
		\Comment{Compute surface given the Car Frame}
		\If{$S_{\mbox{\emph{cur}}} > S_{\mbox{\emph{prev}}}$} 
		\Comment{Search only if the surface increases}
		\State $S_{\mbox{\emph{prev}}} = S_{\mbox{\emph{cur}}}$
		\State \emph{D} = \textsc{filter\_doors}
		($w_{\mbox{\emph{shaft}}}$, \emph{D})
		\Comment{Discard doors that do not fit
			the shaft}
		\State \emph{prototype} = $\emptyset$ \Comment{Empty prototype}
		\For{$j = 1 : D.length$}
		\State \emph{ld} = 
		\textsc{match\_landing\_door}(\emph{D[j]})
		\Comment{Find a matching landing door for $D[j]$}
		\State \emph{prototype} = (\emph{F[i], D[j], ld})
		\If{\textsc{is\_feasible}(\emph{prototype})}
		\Comment{Check whether the tuple is feasible}
		\State \textsc{break}
		\EndIf
		\EndFor
		\If{\emph{prototype} $\neq$ $\emptyset$}
		\State \emph{design} = \textsc{add\_rails}(\emph{prototype})
		\Comment{Add car frame rails to the prototype}
		\State \textsc{optimize\_components}(\emph{design})
		\Comment{Align components}
		\If{\textsc{is\_valid}(\emph{design})}
		\Comment{Validate design after optimization}
		\State \textsc{append}(\emph{output}, \emph{design})
		\EndIf
		\EndIf
		\EndIf
		\EndFor
		\State \Return \emph{output}
		\EndFunction
	\end{algorithmic}
\end{algorithm}

The special-purpose heuristic engine of \liftcreatehr{} is meant to
replicate the design flow that a professional would carry out by hand.
Algorithm~\ref{alg:hr} is a pseudo-code representation of the function
\textsc{compute\_designs} which is at the core of
\liftcreatehr{}. \textsc{compute\_designs} takes as input the shaft 
dimensions $w_{\mbox{\emph{shaft}}}$ and $d_{\mbox{\emph{shaft}}}$,
the list of all available car
doors $D$ and the list of all available car frames $F$. The car door
list is sorted in \emph{decreasing order} according to door size 
(\textsc{order\_by\_opening}) and the car frame list is sorted in
\emph{increasing order} according to the distance between car rails
(\textsc{order\_by\_dcr}). 
The outermost \textbf{for} loop scans the available car frames,
computing for each one the car surface $S_{\mbox{\emph{cur}}}$
(\textsc{compute\_car\_surface}). Here, a heuristic choice is 
performed: if the car surface allowed by the current car frame
$S_{\mbox{\emph{cur}}}$ is not greater than the one allowed by the 
previous one $S_{\mbox{\emph{prev}}}$, the algorithm skips to the next 
choice of car frame. If $S_{\mbox{\emph{cur}}}$ is greater than 
$S_{\mbox{\emph{prev}}}$, the search continues by
filtering the list of car doors (\textsc{filter\_doors}) to discard
those larger than the shaft, while the doors left are scanned in the
innermost \textbf{for} loop --- according to the order of $D$, the
largest opening is considered first to maximize accessibility. For
each door, a matching landing door is selected and 
a \emph{design prototype} is created as the tuple ($F[i], D[j], 
ld$). The predicate \textsc{is\_feasible} checks whether the prototype
is feasible, i.e., the three components can be fitted in the shaft
while respecting all the hard constraints. If so, the innermost loop
comes to an end: implicitly, this is also a heuristic choice since the
procedure does not check other doors that could provide alternative
design prototypes. Given a prototype, the design is finalized by
adding the car frame rails and optimizing the placement with the
function \textsc{optimize\_components}. The predicate
\textsc{is\_valid} checks whether the design is valid and should
be retained for further processing.
Summing up, the procedure produces \textit{at most} one design per
prototype associated with a specific car frame. It is important
to notice that some combinations of car frame and doors that may
result in feasible designs are never explored. While this helps in
keeping the search space at bay, \liftcreatehr{} may return no
designs also in scenarios where feasible ones exist.

\subsection{Genetic algorithms: \liftcreatega{}}
\label{sec:ga_enc}

In \liftcreatega{} the genotype is composed by 4 chromosomes, each one   
consisting of a single gene. As in~\cite{carlson1996genetic}, 
genes are represented by integer numbers as follows:
\begin{itemize}
	\item \emph{Gene 1}: Value of the car door $x$ base point --- $x_{cd}$ 
	in Table~\ref{tab:variables}.
	\item \emph{Gene 2}: Value of the car frame $y$ base point --- $y_{cf}$ 
	in Table~\ref{tab:variables}.
	\item \emph{Gene 3}: Choice of the car frame; to encode the choice
	among available car frames we assigned to each one a unique integer
	identifier.
	\item \emph{Gene 4}: Choice of the car door; also in this case 
	we encoded each door with a unique integer code.
\end{itemize}
Since the available car frame and door identifiers, i.e., the domains of 
the genes 3 and 4, are restricted to those that could fit a given
shaft, each individual represents a single early design. We do not encode
the variables $x_{cf}$ and $y_{cd}$ because, as pointed out in 
Section~\ref{sec:model_constraints}, the coordinates are 
fixed in the design.

The fitness function to score individuals is computed by associating
costs corresponding to violations of the feasibility constraints, 
i.e., we turn hard constraints into soft ones to discourage designs 
that violate them. However, we cannot completely exclude unfeasible designs
and this is why \liftcreatega{} still retains a validation step at the
end to make sure that all generated designs are valid. The cost
function built in \liftcreatega{} includes also terms encoding the
objective presented in Section~\ref{sec:objective} for the choice and 
placement of the car frame and the doors.
We penalize projects in which the car frame is misaligned with respect 
to the car axis --- such as in (\ref{eq:costoh}). If the choice of the car 
frame is such that overhangs are negative --- i.e., if the car frame $dcr$ 
is greater than the resulting car depth $d_{car}$ --- the resulting value 
is multiplied by $10^3$.
Concerning doors, the cost detailed in (\ref{eq:costdoor}) is applied depending
on the door type in order to discourage misaligned placements while for the
selection we apply a slight variation to the hard minimization of (\ref{eq:choice})
by using a parabolic function that grows up quickly for values less than 
550mm or greater than 800mm. The function we consider is 
$f(x) = 0.01 \cdot x^2 - 13.5 \cdot x + 4561.25$ computed with  $x = opening$.
This is because doors whose opening is less than $550$mm and doors with opening 
greater than $800$mm are discouraged: the former are used only in very special
situations and the latter are usually very expensive. Whether the selected door 
is out of range, we apply a fixed weight which is either $10^5$, if the actual 
door width is less than $550$mm, or $300$ if greater than $800$mm.
We use a parabolic function because, as we mentioned before, GAs provide
unconstrained optimization over the solution space, and a
continuous objective is better for the fitness computation.
The GA implementation is provided by the \textit{Jenetics}~\cite{jenetics} 
Java library, which allowed us to seamlessly plug the GA engine to our codebase 
by specifying the composition of the genotype. 

\section{Constraint satisfaction: \liftcreatecp{}/\textsc{SMT}}
\label{sec:selection}

\liftcreatecp{} and \liftcreatesmt{} implement each with their own specific
formalism the encoding detailed in Chapter~\ref{ch:elevator_model}.
Nevertheless, while experimenting in early stages of this encoding and
guided by the experience obtained with search-based techniques we 
identified the following encoding alternatives that can impact the performance
of a constraint solver.

\subsection{Component selection}
The car frame, the cylinder and the doors are selected from a database 
of components. In order to automate the design of an elevator, we must 
consider that choosing different components yields different 
parameter values for each one. The relationship between the selection of 
a component and the assignment of the corresponding parameter 
values can be encoded via Boolean implications of the form
\begin{equation}
	\label{eq:impl}
	Id_x = i \Rightarrow x.p = v
\end{equation}
where $Id_x$ encodes the identifier of choice for component $x$ (a
decision variable), $i$ is a specific identifier value, $x.p$ is some 
parameter of the component $x$ and $v$ is the value of $x.p$ given
that the component $x$ with identifier $i$ was chosen --- see, 
e.g.,~\cite{bacchus2007gac}.
To encode constraints of the form $(\ref{eq:impl})$ a combination
of Boolean reasoning with integer arithmetic is sufficient.
However, considering the way data sets are usually encoded in MiniZinc 
with arrays~\cite{marriott2014minizinc}, we consider an alternative
encoding where we associate an array to each component parameter.
For example, if a component $x$ has two parameters $p_1$ and $p_2$,
we build two arrays $P_1$ and $P_2$ that will store the values of
$p_1$ and $p_2$ for each instance of the component. The index of the
arrays becomes a decision variable chosen by the solver
to enforce the correct values of the parameters.  

\subsection{Look-up tables}
Some parameters, e.g., the maximum number of passengers that the car
may accommodate, are a function of others, e.g., the car
surface. However, instead of expressing such constraints directly --- 
which might involve the use of non-linear or transcendental functions 
--- the correspondence between free parameters and derived ones is 
encoded with \emph{look-up tables}. Table~\ref{tab:passengers} 
exemplifies such a table assuming that the car surface $A$ is contained 
within three ranges.

\begin{table}[t]
	\setlength{\tabcolsep}{20pt}
	\caption{\label{tab:passengers} Look-up table to encode the
		number of passengers $P$ with starting value $P_0$ as a 
		function of the car surface $A$.}
	\centering
	\begin{tabular}{c c}
		\toprule
		\textbf{Car surface ($A$)} & \textbf{No. of passengers ($P$)} \\ 
		\midrule 
		$a1 < A \le a2$ & $P_{0}$ \\ 
		$a2 < A \le a3$ & $P_{0} + 1$ \\ 
		$A > a3$ & $P_{0} + 2$ \\
		\bottomrule
	\end{tabular}
\end{table}

The car payload is computed in a similar way, but,
since the surface ranges are different, we need another set of 
constraints structured in the same way.
These requirements can be easily modeled with implications in the same
way as component selection: the surface $A$ is a decision variable
that implies the number of passengers or the payload. However,
both SMT-LIB and MiniZinc allow users to define custom 
\textit{functions}. In practice, functions are series of 
\textit{if-then-else} statements about, e.g., the car surface, where 
each function returns, e.g., the corresponding number of passengers 
or the payload.

\subsection{Integers vs. Reals}
Most parameters involved in the design process for elevators are 
expressed in millimeters which suggests integer-based encodings.
However, some parameters, like the forces exerted on the car rails,
involve arithmetic over reals. This makes the corresponding
constraint satisfaction problems members of the mixed-integer
arithmetic family. In such encodings, the main disadvantage is that a 
large number of integer quantities may increase considerably the
solution time. We try to improve on this by relaxing some of the
integer quantities to reals.
In particular, we consider component parameters since 
parameters are not \textit{decided} but their values are only assigned
based on the choice of a component. This means that the domain of the
parameters is a finite set and we can relax the arithmetic encoding 
without producing invalid results. In this representation the only 
operation that could add decimal digits is division, but since in our
encoding there are only a few such operations, boundary checking can 
be implemented easily. These considerations do not hold for some 
decision variables including, but not limited to, the index used to 
select components. Also, several CP solvers we experimented with are 
not affected by this choice due to the fact that they do not support
floating-point arithmetic.

\subsection{Single and Multi-objective optimization}

Here we describe alternative constructions of the cost function,
using the contributions that are detailed in Section~\ref{sec:objective}.
In previous works of ours~\cite{AiLift2} we consider the weighted sum 
of the costs $c_{cf}$, $c_{door}$ and $c_{size}$ to obtain the overall
cost function, but the contribution $c_{cr}$ may conflict with the
previous ones because the farthest is the door from the car frame, the
greater is the force exerted on the car rails~\cite{AiLift3, AiLift4}. 
Nevertheless, since the car rails can be chosen once the other components
are fitted, this objective can be considered with a lower priority. 
If we follow a single-objective approach, we can weight the cost $c_{cr}$
significantly less than the other three. The overall cost function
$\mathcal{C}$ becomes
\begin{equation}
	\label{eq:costfunc_single}
	\mathcal{C} = \alpha_1 c_{cf} + \alpha_2 c_{door} +  \alpha_3 c_{size}
	+ \alpha_4 c_{cr} 
\end{equation}
with $\alpha_4 \ll \alpha_i$ for $i \neq 4$.

Alternatively, we can exploit priorities among different cost
functions by resorting to multi-objective optimization using, e.g.,
the lexicographic method whereby preferences are imposed by ordering 
the objective functions according to their significance ---
see~\cite{CHANG20151105} for details.
In this case, we consider two different cost functions:
\begin{equation}
	\label{eq:costfunc_multi}
	\begin{array}{rcl}
		\mathcal{C}_1 & = & \alpha_1 c_{cf} + \alpha_2 c_{door} +
		\alpha_3 c_{size}\\
		\mathcal{C}_2 & = &c_{cr}
	\end{array}
\end{equation}
where the objective function $\mathcal{C}_1$ is minimized first.

\liftcreatecp{} and \liftcreatesmt{} take into account the constraints 
and the cost function providing an encoding to a formula in the 
MiniZinc~\cite{nethercote2007minizinc} and SMT-LIB~\cite{smtlib} 
languages, respectively, to be solved by a number of state-of-the-art 
solvers. 
%Considering the optimization, the cost function defined in 
%(\ref{eq:costfunc_single}) is applied with 
%$\alpha_1 = \alpha_2 = \alpha_3  = 1$ and $\alpha_4 = 0.1$ 
%i.e., a setup where every configuration objective in $\mathcal{C}_1$ 
%has the same priority.
