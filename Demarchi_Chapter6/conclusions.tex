\section{Conclusions}
\label{sec:concl}
%
The work presented in this Thesis has advanced the state of the art with the
following contributions:
%
\begin{itemize}
	\item We experimented with declarative encodings for the automated design
		of a complex system, and generalized some best practices that can
		apply to different case studies.
	\item We expanded and structured the reachability analysis for DNNs employing
		the notion of generalized Star set.
	\item We developed some variations of the verification algorithms pushing
		towards the optimization of some existing bottlenecks in the procedures.
	\item We developed state of the art tools for both case studies, i.e., a
		web-based application for the design of elevator systems and two GUIs
		for the construction, conversion, learning and verification of DNNs.
\end{itemize}
%
In~\cite{AiLift2} we started by encoding the elevator design problem first described
in~\cite{AiLift} using the Satisfiability Modulo Theories paradigm, and we observed that
with the employ of specific constraints it was possible to reach and also outperform the
baseline algorithm. Encouraged by this result, we considered using pure Constraint 
Programming solvers in~\cite{AiLift3} in order to understand the impact of the encoding
choiches that we observed with SMT. While a full encoding is still best handled by an
SMT solver when encountering computations involving real parameters, CP solvers provided
a significant speed-up when considering integer encodings only. Finally, in~\cite{AiLift4}
we resumed all our work and included another comparison using Genetic Algorithms.

We applied the work of~\cite{DBLP:conf/ecai/GuidottiLPT20} and~\cite{guidotti2021pynever} 
to new case studies, namely Adaptive Cruise Control in~\cite{DBLP:conf/ecms/DemarchiGPT22} 
and Reinforcement Learning-based drone control. Here we perfectioned our tool \nevertwo{} 
in order to build the networks, train them and define the properties to verify. The 
drone setup is the first step for a RL-based verification framework for robotics case 
studies that we are building. We also investigated an optimization for parts of our 
verification methodology in order to cope with the pervasive and well-known scalability 
issues: in~\cite{DBLP:conf/cpsschool/DemarchiG22} we investigated a counter-example guided
abstract refinement which was introduced in~\cite{DBLP:phd/basesearch/Guidotti22} and we 
experimented with a way for deleting degenerate stars in the complete verification algorithm.

Finally, we published state of the art tools in both domains. \liftcreate{} is a research
prototype available at \url{liftcreate.ailift.it}, still focused on the heuristic engine 
but designed in order to be able to switch between new engines on the fly.
\coconet{} and \nevertwo{} are the two interfaces in the NeVerTools portfolio, which is 
part of a larger open-source ecosystem which is still in progress, called
\textit{NeuralVerification}\footnote{www.neuralverification.org}: here we collect 
all the resources about our research and development including the VNN-LIB 
standard~\cite{vnnlib} that we use in the verification community.

In conclusion, the work presented in this Thesis can be evaluated on the initial research
questions:
\begin{itemize}
	\item $(i)$ The choice of specific encodings, i.e., arithmetic theories, logic
	encapsulation, optimization, is a key point in the design encoding. In fact, the
	experimental evaluation proves how impactful they can be in order to obtain a
	result faster than other methodologies.
	
	\item $(ii)$ The choice of solvers and tools is complementary to the choice of the
	encoding: there is no silver bullet when it comes to choose a tool for solving a 
	design problem. Depending on the problem shape and how it is encoded, different
	solvers yield different results.
	
	\item $(iii)$ The integration of new encodings in an existing and complex application
	was a challenging task that impacted the original architecture as well. In fact, in
	order to factorize common elements in the design flow and to build design processes
	that could be seamlessly integrated in the base application, it was necessary to rewrite
	or refactor several tasks that degraded the baseline performances, too.
	
	\item $(iv)$ Abstraction-based methods for the verification of neural networks have
	proven very efficient in the compact representation of sets for evaluating the 
	reachability of a network. On the other hand, exact analysis suffers from the curse
	of dimensionality when it is necessary to branch into several alternatives in order
	to explore exhaustively the solutions space. Corroborated by different case studies,
	this Thesis proposes an improvement of exact reachability analysis by discarding
	duplicate solutions during branching.
	
	\item $(v)$ The most important contribution that is presented in this Thesis is the
	\textit{NeVerTools} suite with the tools \coconet{} and \nevertwo. In fact, \nevertwo{}
	is the only tool in the verification community to provide a single environment where it
	is possible to build, edit, learn and verify a network. Leveraging the VNN competition,
	the employ of \coconet{} and \nevertwo{} by practitioners interested in providing
	guarantees on their systems should benefit from contributions by the whole community.
\end{itemize}