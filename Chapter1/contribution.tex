\section{Contribution}
\label{sec:contribution}

The contribution of this Thesis is twofold. First, the results of the
experiments on different ways of encoding design problems and how to
employ constraint satisfaction techniques in verification of NNs contribute
to deepen the knowledge in how different ways of encoding the constraints
impact on the results. Second, for both case studies we developed special
purpose tools that provide access to all the methodologies taken in exam.
In particular, we contribute with the following results:

\paragraph{Declarative encodings for elevator systems.}
Starting from the work presented in~\cite{AiLift2} we investigated how to
benefit from declarative encodings in the elevator domain, and how to
transfer such knowledge to generic configuration systems. In~\cite{AiLift3}
and~\cite{AiLift4} we experimented with several declarative encodings based
on Genetic Algorithms (GAs), Satisfiability Modulo Theories (SMT) and 
Constraint Programming (CP) showing the strength and weaknesses of these
approaches against the different aspects of the design task.

\paragraph{Neural networks verification in safety-critical domains.}
We use SMT as a pivot to deal with the verification of machine learning 
models with neural networks. We created a standard for the definition
of benchmarks which uses SMT as the language for the definition of the
verification properties~\cite{vnnlib} and we contribute with two new case
studies, i.e., Adaptive Cruise Control (ACC)~\cite{DBLP:conf/ecms/DemarchiGPT22}
and drone hovering. We also experiment with optimization algorithms for
our verification procedure~\cite{DBLP:conf/cpsschool/DemarchiG22}.

\paragraph{State of the art tool development.}
Alongside the encodings and the experiments herewith reported, we also
developed new tools for providing fast and easy access to the algorithms.
For the design of elevator systems we perfectioned the tool \liftcreate,
which provides an interface towards the different encodings. Using this
tool, the experiments shared the same overhead and provided reliable and
replicable results.

In the verification topic, we created a software portfolio named
\textit{NeVerTools}\footnote{https://github.com/NeVerTools} which groups
the former contributions in the domain, packed in the \pynever{} Python API
and two Graphical User Interfaces (GUIs), namely \coconet{} and \nevertwo.