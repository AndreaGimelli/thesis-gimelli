\section{Constraint Satisfaction Problems}

A Constraint Satisfaction Problem (CSP) requires a value, selected 
from a given finite domain, to be assigned to each variable in the 
problem, so that all constraints relating the variables are
satisfied~\cite{brailsford1999constraint}. In more detail, given a 
set of variables together with their domains, i.e., a set of possible 
values that can be assigned to each variable, and a description of 
the problem in the form of mathematical constraints, a CSP is the 
problem of finding values of the variables that satisfy every 
constraint. It is defined as a set of $n$ \textit{variables} 
$X = \{x_1, \ldots, x_n\}$, a set of current \textit{domains}
$\mathcal{D} = \{D(x_1), \ldots, D(x_n)\}$ where $D(x_i)$ is the
finite set of possible \textit{values} for variable $x_i$, and a 
set $\mathcal{C}$ of \textit{constraints} between variables.
A constraint $C$ on the set of variables $X(C) = (x_{i_1}, \ldots, 
x_{i_r})$ is a subset of the Cartesian product $D_0(x_{i_1}) \times 
\ldots \times D_0(x_{i_r})$ that specifies the \textit{allowed}
combinations of values for the variables $x_{i_1} \times \ldots
\times x_{i_r}$.
 
In this Thesis we consider two main approaches to solve CSPs, 
namely Constraint Programming (CP) and Satisfiability
Modulo Theory (SMT). CP is a powerful paradigm for solving CSPs
that draws on a wide range of techniques from artificial 
intelligence, operations research, algorithms, graph theory and 
others~\cite{rossi2006handbook}. In general, CP provides high level
languages that allow one to describe (model) a problem in a 
declarative way by means of constraints, that is, properties of the
solutions to be found. Product configuration has been established 
as a successful application area of CP~\cite{mcdonald2002case,
	jensen2004clab, benavides2005using, falkner2011modeling,
	hervieu2016practical}. A CP solver  checks whether a certain
assignment of decision variables respects all the constraints, 
and returns such assignment in that case. Most CP solvers deal 
naturally with finite domains and have good propagators to handle
injectivity constraints. SMT handles the original problem by 
encoding it to the problem of deciding the satisfiability of a 
first-order formula with respect to some decidable theory
$\mathcal{T}$. In particular, SMT generalizes the Boolean 
satisfiability problem (SAT) by adding background theories such 
as the theory of real numbers, the theory of integers, and the 
theories of data structures (\textit{e.g.}, lists, arrays and 
bit vectors) --- see, e.g.,~\cite{DBLP:series/faia/BarrettSST09}
for details. There exists a constraint modeling language for 
designing constraint satisfaction and optimization problems in 
a high-level, solver independent way called 
MiniZinc~\cite{marriott2014minizinc}. SMT solvers comply instead 
to a dedicated standard language called SMT-LIB~\cite{smtlib}.

Here we provide some more detail about the SMT approach which we use
as a basis to introduce our modeling of design and configuration for
RHEs. The corresponding CP formulation can be obtained with a simple
syntax-driven translation. To decide the satisfiability of an input
formula $\varphi$ in conjunctive normal form, SMT solvers typically
first build a \emph{Boolean  abstraction} $\textit{abs}(\varphi)$ of
$\varphi$ by replacing each constraint by a fresh Boolean variable
(proposition), e.g., 
\begin{eqnarray*}
	\arraycolsep=2pt
	\begin{array}{ccccccccccc}
		\varphi &: &\underbrace{x \geq y} &\wedge &(&\underbrace{y > 0}& \vee &\underbrace{x >0}&)& \wedge &\underbrace{y \leq 0} \\
		\textit{abs}(\varphi)&:&A& \wedge& (&B& \vee &C&)& \wedge  &\neg B
	\end{array}
	\label{eq:abs}
\end{eqnarray*}
where $x$ and $y$ are real-valued variables, and $A$, $B$ and $C$ are
propositions. A propositional logic solver searches for a satisfying
assignment $S$ for $\textit{abs}(\varphi)$, e.g., $S(A)=1$, $S(B)=0$, 
$S(C)=1$ for the above example.  If no such assignment exists then the
input formula $\varphi$ is unsatisfiable. Otherwise, the consistency
of the assignment in the underlying theory is checked by a
\emph{theory solver}. In our example, we check whether the set $\{ x
\geq y,\ y \leq 0,\ x > 0\}$ of linear inequalities is feasible, which
is the case. If the constraints are consistent then a satisfying
solution (\textit{model}) is found for $\varphi$. Otherwise, the
theory solver returns a theory lemma $\varphi_E$ giving an
\textit{explanation} for the conflict, e.g., the negated conjunction
some inconsistent input constraints. 
The explanation is used to refine the Boolean abstraction
$\textit{abs}(\varphi)$ to $\textit{abs}(\varphi)\wedge
\textit{abs}(\varphi_E)$. These steps are iteratively executed until
either a theory-consistent Boolean assignment is found, or no more
Boolean satisfying assignments exist.

Adding theories of cost to SMT yields Optimization Modulo Theories
(OMT), an extension that finds models
to optimize given objectives through a combination of SMT and optimization 
procedures~\cite{sebastiani2012optimization}. For example,
\begin{equation*}
	\left\{
	\begin{array}{l}
		\varphi : x \geq y \wedge (y > 0 \vee x > 0) \wedge y \leq 0 \\
		\min_{x,y} ( x + y )
	\end{array}
	\right.
\end{equation*}
requires all the constraints in $\varphi$ to be satisfied
and the additional cost $x + y$ to be minimized.
Notice that OMT extends classical formulations  
in mathematical programming, e.g., linear programming or mixed integer
linear programming, since it allows Boolean structure to be taken into
account  together with the optimization target. OMT solvers have been
developed for several first-order theories like, e.g., those of
linear arithmetic over the rationals $(LRA)$ or the integers
$(LIA)$ or their combination $(LIRA)$. In this work we mainly
consider \textit{quantifier free} theories in a mixed 
integer/rational domain --- known as $QF\_LIRA$ in the
literature~\cite{barrett2018satisfiability}.