\section{Genetic algorithms}
\label{sec:gas}

Genetic Algorithms (GAs) are optimization procedures based on ideas 
borrowed from natural selection and evolution. Detailed descriptions
of GAs are to be found, e.g., in~\cite{davis1991handbook}. An
application of GAs to a relatively simple automated configuration
problem together with a comparison with other declarative techniques
can be traced back to~\cite{falkner2011modeling}, and
other earlier references leveraging GAs for automated product
configuration can be found in~\cite{zhang2014product}. A recent
survey~\cite{DBLP:journals/nca/SlowikK20} cites many different
applications of GAs and other evolutionary algorithms to engineering
problems, including automated configuration and optimization scenarios
such as, e.g., optimization of solar array layouts~\cite{lv2017solar}, load 
balancing in cargo ships~\cite{ramos2018new}, and building
energy-efficient houses~\cite{ascione2016simulation}.  
For the purpose of this paper, it is sufficient to recall that GAs
consider a \emph{population} as a finite set $P$ of potential
solutions to the target optimization problem. Each individual $p \in
P$ is characterized by a \emph{genotype} comprised of
\emph{chromosomes}. As in nature, chromosomes define the individual
and are the basis for the obtaining different individuals by
``mating'' procedures. The \emph{fitness function} is a mapping $f : P \to
\mathbb{R}$ which  ranks the individuals according to a \emph{fitness
  score}: the higher the chance of being a good solution, the higher the
fitness  score. Notice that GAs provide \emph{unconstrained
  optimization} over the space of potential solutions. 
In order to take into account constraints, as our elevator design problem 
requires, the fitness function should contain one or more \emph{loss 
	factors} --- see, e.g.,~\cite{homaifar1994constrained, 
	yeniay2005penalty, gungor2022meta} --- which 
penalize the individual design when it violates specific constraints: 
in this way, hard constraints are turned into preferences about 
solutions. By shaping the loss factors adequately we are able to control 
how much getting closer to violating a constraint can be discouraged.  
GAs are initialized with a randomly chosen population $P$ and then
they seek to improve the initial choice by repeating the
following steps: 
\begin{enumerate}
	\item the fitness $f(p)$ of each individual $p \in P$ is computed; 
	\item a set $M \subset P$ is extracted from $P$ such that individuals
	in $M$ have the highest fitness among those in $P$;
	\item the individuals in $M$ are subject to ``mating'' procedures
	such as \emph{crossover}, or other evolutionary phenomena such as
	\emph{mutation}: informally, crossover occurs when the
        genotypes of two individuals are split and recombined to form
        new ones bearing some chromosomes, i.e., common traits, from
        both their parents. 
	\item The result of the previous step is a population $P'$ which might
	contain individuals fitter than those of the
	previous population $P$; in  particular, the crossover operation
	attempts to combine the genes of fit individuals to produce fitter
	children, and mutation attempts to maintain diversity
	in a population of designs. 
	\item Population $P'$ becomes the new population $P$ and the search
	restarts from step (1) unless some \emph{termination condition}
	occurs, e.g., the fitness of the fittest individuals did not
	change in the last $k$ steps, or a fixed number of $h$ generations has
	been produced, where $k$ and $h$ are user-controlled hyper-parameters. 
\end{enumerate}
%In our case, each element of the population is a RHE design, and the
%genotype is meant to describe its  main elements.